基于\+C++的神经网络算法实现


\begin{DoxyItemize}
\item \mbox{\hyperlink{class_matrix}{Matrix}}

\quad{}\quad{}是为实现神经网络实现的矩阵类,用于作为存储网络参数的数据结构。
\item activations

\quad{}\quad{}激活函数类,实现常见激活函数
\item 魔数

\quad{}\quad{}很多类型的文件,其起始的几个字节的内容是固定的(或是有意填充,或是本就如此)。根据这几个字节的内容就可以确定文件类型,因此这几个字节的内容被称为魔数 (magic number)。
\item 大端存储与小端存储

\quad{}\quad{}大端存储类似人的正常思维,小端存储机器处理更方便。 \begin{quote}
小端存储(\+Little-\/\+Endian):数据的高字节存储在高地址中,数据的低字节存储在低地址中 。 \end{quote}
\begin{quote}
大端存储(\+Big-\/\+Endian):数据的高字节存储在低地址中,数据的低字节存储在高地址中。socket编程中网络字节序一般是大端存储 \end{quote}

\begin{DoxyCode}{0}
\DoxyCodeLine{\{C++\}}
\DoxyCodeLine{ int i = 1;}
\DoxyCodeLine{ // int在内存中占 4Byte; }
\DoxyCodeLine{ // a在内存中的分布为 0x1 0x0 0x0 0x0; }
\DoxyCodeLine{ // 从左到右内存地址降低,也就是高字节地址存放的是a的低字节数据}

\end{DoxyCode}
 \quad{}\quad{}mnist原始数据文件中32位的整型值是大端存储,\+C/\+C++变量是小端存储,所以读取数据的时候,需要对其进行大小端转换。 (注意:只有文件头的个别数字需要大小端转换,其余的60000个有效数据则不需要) 
\end{DoxyItemize}