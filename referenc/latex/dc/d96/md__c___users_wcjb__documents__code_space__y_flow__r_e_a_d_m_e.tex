\quad{}\quad{}基于\+C++和cuda实现的深度学习框架\+YFlow,这是作者练手之作,用于总结回顾平时工作中各类算法,当然,由于局限于个人水平,目前仍有许多不足之处,目前计划是在一年内完成深度学习相关开发底层实用\+C++和\+CUDA混合开发。在实现过程中,我本人也接触学习了许多底层算法计算优化的技巧,本项目参考了\+Darknet框架的内容,并尝试进行突破。本项目的最终目标是成为一个基于\+C++和\+CUDA实现的高效率机器学习框架。


\begin{DoxyItemize}
\item 项目结构
\begin{DoxyItemize}
\item Random

\quad{}\quad{}随机数模块,用于生成符合统计分布的随机数,基于学习理论的考虑,会依次实现当下所有随机数生成算法,实现进度如下\+:
\begin{DoxyItemize}
\item \mbox{[}x\mbox{]} 冯·诺伊曼平方取中法及其改良算法;
\item \mbox{[}x\mbox{]} 线性同余发生器(Kobayashi混合同余发生器);
\item \mbox{[} \mbox{]} 梅森旋转演算法;
\item \mbox{[} \mbox{]} FSR发生器;
\item \mbox{[} \mbox{]} 组合发生器(因为效率原因,未作实现);
\end{DoxyItemize}

\quad{}\quad{}上述随机数生成算法用于生成符合均匀分布(即一个周期内所有数据出现是等概率)的随机数,再基于均匀分布生成其它分布的随机数。实现进度如下:
\begin{DoxyItemize}
\item \mbox{[}x\mbox{]} 均匀分布
\item \mbox{[} \mbox{]} 正态分布(高斯分布)
\item \mbox{[} \mbox{]} 二项分布
\item \mbox{[} \mbox{]} 伯努利分布
\item \mbox{[} \mbox{]} 拉普拉斯分布
\item \mbox{[} \mbox{]} 泊松分布
\item \mbox{[} \mbox{]} 指数分布
\item \mbox{[} \mbox{]} 伽马分布
\item \mbox{[} \mbox{]} 贝塔分布
\item \mbox{[} \mbox{]} 狄拉克分布
\end{DoxyItemize}
\item tools

\quad{}\quad{}深度学习常用基本初等函数的实现,目前已完成的实现有:
\begin{DoxyItemize}
\item \mbox{[}x\mbox{]} 快速幂
\item \mbox{[}x\mbox{]} 卡马克快速倒数平方根算法
\item \mbox{[}x\mbox{]} 基于泰勒级数和\+Remez算法的自然对数函数(ln)的实现
\item \mbox{[} \mbox{]} 基于积分变上限函数和自适应辛普森法的自然对数函数(ln)的实现
\end{DoxyItemize}
\end{DoxyItemize}
\end{DoxyItemize}

\begin{DoxyVerb}+ Tensor

    &emsp;&emsp;实现张量数据结构,是构建所有算法模型最基本单元,实现0~3阶张量。
\end{DoxyVerb}
 