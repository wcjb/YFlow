基于\+C++和\+CUDA的神经网络框架实现


\begin{DoxyItemize}
\item 魔数

\quad{}\quad{}很多类型的文件,其起始的几个字节的内容是固定的(或是有意填充,或是本就如此)。根据这几个字节的内容就可以确定文件类型,因此这几个字节的内容被称为魔数 (magic number)。
\item 大端存储与小端存储

\quad{}\quad{}大端存储类似人的正常思维,小端存储机器处理更方便。 \begin{quote}
小端存储(\+Little-\/\+Endian):数据的高字节存储在高地址中,数据的低字节存储在低地址中 。 \end{quote}
\begin{quote}
大端存储(\+Big-\/\+Endian):数据的高字节存储在低地址中,数据的低字节存储在高地址中。socket编程中网络字节序一般是大端存储 \end{quote}

\begin{DoxyCode}{0}
\DoxyCodeLine{\{C++\}}
\DoxyCodeLine{ int i = 1;}
\DoxyCodeLine{ // int在内存中占 4Byte; }
\DoxyCodeLine{ // a在内存中的分布为 0x1 0x0 0x0 0x0; }
\DoxyCodeLine{ // 从左到右内存地址降低,也就是高字节地址存放的是a的低字节数据}

\end{DoxyCode}
 \quad{}\quad{}mnist原始数据文件中32位的整型值是大端存储,\+C/\+C++变量是小端存储,所以读取数据的时候,需要对其进行大小端转换。 (注意:只有文件头的个别数字需要大小端转换,其余的60000个有效数据则不需要)
\item 牛顿迭代算法

\quad{}\quad{}将\$f(x)\$在\$x\+\_\+0\$处进行泰勒展开有: \$\$ f(x) = \textbackslash{}sum\+\_\+\{n = 0\}$^\wedge$\{\textbackslash{}infty\}\textbackslash{}frac\{f$^\wedge$\{n\}(x\+\_\+0)\}\{n!\}(x-\/x\+\_\+0)$^\wedge$n \$\$ 取泰勒展开的前两项作为\$f(x)\$的近似,则有: \$\$ f(x) = f(x\+\_\+0)+f\{\textquotesingle{}\}(x\+\_\+0)(x-\/x\+\_\+0) \$\$ 现在要求\$f(x)\$的根,则有: \$\$ f(x) =0 \textbackslash{}to x = x\+\_\+0+\textbackslash{}frac\{f(x\+\_\+0)\}\{f$^\wedge$\{\textquotesingle{}\}(x\+\_\+0)\} \$\$ 于是得到了方程根的近似值,并且可以证明,随着不断按下式进行迭代后可以以任意精度逼近方程\$f(x)=0\$根: \$\$ f(x\+\_\+\{n+1\}) = f(x\+\_\+n)-\/\textbackslash{}frac\{f(x\+\_\+n)\}\{f$^\wedge$\{\textquotesingle{}\}(x\+\_\+n)\} \$\$ 我们称这个迭代算法为牛顿迭代算法,其本质思想{\ttfamily 是用函数的切线来对函数进行任意点的逼近}。 \begin{quote}
以求\$a\$的平方根倒数为例\+: \end{quote}
\quad{}\quad{}求\$a\$的平方根倒数 \$\$ x = \textbackslash{}frac\{1\}\{\textbackslash{}sqrt\{a\}\} \$\$ 等价于求方程 \$\$ \textbackslash{}frac\{1\}\{x$^\wedge$2\}-\/a=0 \$\$ 的根,令\+: \$\$ f(x) = \textbackslash{}frac\{1\}\{x$^\wedge$2\} -\/a \$\$ 则有: \$\$ f$^\wedge$\{\textquotesingle{}\}(x) = \textbackslash{}frac\{-\/2\}\{x$^\wedge$3\} \$\$ 得到牛顿迭代式为\+: \$\$ x\+\_\+\{n+1\} = \textbackslash{}frac\{x\+\_\+n\}\{2\}(3-\/ax\+\_\+n$^\wedge$2) \$\$ 再结合神奇的$\ast$$\ast$魔术数字$\ast$$\ast$即可得到大名鼎鼎的快速平方根倒数算法: 
\begin{DoxyCode}{0}
\DoxyCodeLine{\{C++\}}
\DoxyCodeLine{ float InvSqrt (float x)}
\DoxyCodeLine{ \{}
\DoxyCodeLine{     float xhalf = 0.5f*x;}
\DoxyCodeLine{     int i = *(int*)\&x;}
\DoxyCodeLine{     i = 0x5f3759df -\/ (i>>1);}
\DoxyCodeLine{     x = *(float*)\&i;}
\DoxyCodeLine{     x = x*(1.5f -\/ xhalf*x*x);}
\DoxyCodeLine{     return x;}
\DoxyCodeLine{ \}}

\end{DoxyCode}
 
\end{DoxyItemize}